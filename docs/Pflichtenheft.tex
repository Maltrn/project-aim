\documentclass[a4paper,12pt]{article}
\usepackage{amssymb} % needed for math
\usepackage{amsmath} % needed for math
\usepackage[utf8]{inputenc} % this is needed for german umlauts
\usepackage[ngerman]{babel} % this is needed for german umlauts
\usepackage[T1]{fontenc}    % this is needed for correct output of umlauts in pdf
\usepackage[margin=2.5cm]{geometry} %layout
\usepackage{booktabs}

% this is needed for forms and links within the text
\usepackage{hyperref}  

% glossar, see http://en.wikibooks.org/wiki/LaTeX/Glossary
% has to be loaded AFTER hyperref so that entries are clickable
\usepackage[nonumberlist]{glossaries} 

% The following is needed in order to make the code compatible
% with both latex/dvips and pdflatex.
\ifx\pdftexversion\undefined
\usepackage[dvips]{graphicx}
\else
\usepackage[pdftex]{graphicx}
\DeclareGraphicsRule{*}{mps}{*}{}
\fi

\makeglossary 

%%%%%%%%%%%%%%%%%%%%%%%%%%%%%%%%%%%%%%%%%%%%%%%%%%%%%%%%%%%%%%%%%%%%%%
% Variablen                                 						 %
%%%%%%%%%%%%%%%%%%%%%%%%%%%%%%%%%%%%%%%%%%%%%%%%%%%%%%%%%%%%%%%%%%%%%%
\newcommand{\authorName}{Dustin Spallek, Rene Kretschmer, Christof Rode, Jana Wengenroth, Malte Scheller, Daniel Schruhl}
\newcommand{\projektName}{Microservices mit Angular}
\newcommand{\tags}{\authorName, Lastenheft}
\newcommand{\glossarName}{Glossar}
\title{\projektName~(Pflichtenheft)}
\author{\authorName}
\date{\today}

%%%%%%%%%%%%%%%%%%%%%%%%%%%%%%%%%%%%%%%%%%%%%%%%%%%%%%%%%%%%%%%%%%%%%%
% PDF Meta information                                 				 %
%%%%%%%%%%%%%%%%%%%%%%%%%%%%%%%%%%%%%%%%%%%%%%%%%%%%%%%%%%%%%%%%%%%%%%
\hypersetup{
  pdfauthor   = {\authorName},
  pdfkeywords = {\tags},
  pdftitle    = {\projektName~(Pflichtenheft)}
} 
 
%%%%%%%%%%%%%%%%%%%%%%%%%%%%%%%%%%%%%%%%%%%%%%%%%%%%%%%%%%%%%%%%%%%%%%
% Create a shorter version for tables. DO NOT CHANGE               	 %
%%%%%%%%%%%%%%%%%%%%%%%%%%%%%%%%%%%%%%%%%%%%%%%%%%%%%%%%%%%%%%%%%%%%%%
\newcommand\addrow[2]{#1 &#2\\ }

\newcommand\addheading[2]{#1 &#2\\ \hline}
\newcommand\tabularhead{\begin{tabular}{lp{13cm}}
\hline
}

\newcommand\addmulrow[2]{ \begin{minipage}[t][][t]{2.5cm}#1\end{minipage}% 
   &\begin{minipage}[t][][t]{8cm}
    \begin{enumerate} #2   \end{enumerate}
    \end{minipage}\\ }

\newenvironment{usecase}{\tabularhead}
{\hline\end{tabular}}




%%%%%%%%%%%%%%%%%%%%%%%%%%%%%%%%%%%%%%%%%%%%%%%%%%%%%%%%%%%%%%%%%%%%%%
% THE DOCUMENT BEGINS             	                              	 %
%%%%%%%%%%%%%%%%%%%%%%%%%%%%%%%%%%%%%%%%%%%%%%%%%%%%%%%%%%%%%%%%%%%%%%
\begin{document}
 \pagenumbering{roman}
 \begin{titlepage}
\maketitle
\thispagestyle{empty} % no page number

\begin{verbatim}












\end{verbatim}


  \begin{tabular}[t]{ll}
	Projekt:       & \quad \projektName \\[1.2ex]
  \end{tabular}

\begin{tabular}{|p{3 cm}|p{3 cm}|p{5 cm}|}
\hline
\textbf{Version} & \textbf{Datum} & \textbf{Autor(en)} \\
\hline
\hline
1.0 & 19.09.2016 & \authorName \\
\hline
\end{tabular}
\end{titlepage}
         % Deckblatt.tex laden und einfügen
 \setcounter{page}{2}
 \tableofcontents          % Inhaltsverzeichnis ausgeben
 \clearpage
 \pagenumbering{arabic}
 
\section{Zielbestimmung}
\subsection{Musskriterien}
%%%%%%%%%%%%%%%%%%%%%%%%%%%%%%%%%%%%%%%%%%%%%%%%%%%%%%%%%%%%%%%%%%%%%%
% Was muss das Programm können?                   					 %
%%%%%%%%%%%%%%%%%%%%%%%%%%%%%%%%%%%%%%%%%%%%%%%%%%%%%%%%%%%%%%%%%%%%%%
\begin{usecase}
  \addheading{Nummer}{Beschreibung} 
  \addrow{/FA10/}{Lorem ipsum dolor sit amet, consetetur sadipscing elitr, sed diam}
  \addrow{/FA20/}{Lorem ipsum dolor sit amet, consetetur sadipscing elitr}
  \addrow{/FA30/}{adasdfasdf asdf asdfas}
\end{usecase}

\subsection{Wunschkriterien}
%%%%%%%%%%%%%%%%%%%%%%%%%%%%%%%%%%%%%%%%%%%%%%%%%%%%%%%%%%%%%%%%%%%%%%
% Was muss das Programm können?                   					 %
%%%%%%%%%%%%%%%%%%%%%%%%%%%%%%%%%%%%%%%%%%%%%%%%%%%%%%%%%%%%%%%%%%%%%%
\begin{usecase}
  \addheading{Nummer}{Beschreibung} 
  \addrow{/FA10/}{Lorem ipsum dolor sit amet, consetetur sadipscing elitr, sed diam}
  \addrow{/FA20/}{Lorem ipsum dolor sit amet, consetetur sadipscing elitr}
  \addrow{/FA30/}{adasdfasdf asdf asdfas}
\end{usecase}

\subsection{Abgrenzungskriterien}
%%%%%%%%%%%%%%%%%%%%%%%%%%%%%%%%%%%%%%%%%%%%%%%%%%%%%%%%%%%%%%%%%%%%%%
% Was muss das Programm können?                   					 %
%%%%%%%%%%%%%%%%%%%%%%%%%%%%%%%%%%%%%%%%%%%%%%%%%%%%%%%%%%%%%%%%%%%%%%
\begin{usecase}
  \addheading{Nummer}{Beschreibung} 
  \addrow{/FA10/}{Lorem ipsum dolor sit amet, consetetur sadipscing elitr, sed diam}
  \addrow{/FA20/}{Lorem ipsum dolor sit amet, consetetur sadipscing elitr}
  \addrow{/FA30/}{adasdfasdf asdf asdfas}
\end{usecase}


\section{Produkteinsatz}
%%%%%%%%%%%%%%%%%%%%%%%%%%%%%%%%%%%%%%%%%%%%%%%%%%%%%%%%%%%%%%%%%%%%%%
% Wer ist die Zielgruppe?                   						 %
%%%%%%%%%%%%%%%%%%%%%%%%%%%%%%%%%%%%%%%%%%%%%%%%%%%%%%%%%%%%%%%%%%%%%%
Der Microservice soll in die Infrastruktur von Symphony eingegliedert werden, um für Anbieter die Funktion der Anreicherung ihrer Produktdaten zu ermöglichen.

\subsection{Zielgruppe}
Unsere Zielgruppe sind alle Mitarbeiter der Anbieter, die einen Zugang zum System haben.

\subsection{Anwendungsbereiche}
Administrative Anwendungsbereiche zur Verwaltung und Anreicherung der Anbieter und Produktdaten.

\section{Funktionale Anforderungen}
%%%%%%%%%%%%%%%%%%%%%%%%%%%%%%%%%%%%%%%%%%%%%%%%%%%%%%%%%%%%%%%%%%%%%%
% Was muss das Programm können?                   					 %
%%%%%%%%%%%%%%%%%%%%%%%%%%%%%%%%%%%%%%%%%%%%%%%%%%%%%%%%%%%%%%%%%%%%%%
\begin{usecase}
  \addheading{Nummer}{Beschreibung} 
  \addrow{/FA10/}{Lorem ipsum dolor sit amet, consetetur sadipscing elitr, sed diam}
  \addrow{/FA20/}{Lorem ipsum dolor sit amet, consetetur sadipscing elitr}
  \addrow{/FA30/}{adasdfasdf asdf asdfas}
\end{usecase}

\section{Produktdaten}
%%%%%%%%%%%%%%%%%%%%%%%%%%%%%%%%%%%%%%%%%%%%%%%%%%%%%%%%%%%%%%%%%%%%%%
% Auf welchen Daten arbeitet das Produkt?                            %
%%%%%%%%%%%%%%%%%%%%%%%%%%%%%%%%%%%%%%%%%%%%%%%%%%%%%%%%%%%%%%%%%%%%%%
\begin{usecase}
  \addheading{Nummer}{Beschreibung} 
  \addrow{/PD10/}{Produkte haben einen Namen}
  \addrow{/PD20/}{Produkte haben eine Kurzbeschreibung}
  \addrow{/PD30/}{Produkte haben eine Detailbeschreibung}
  \addrow{/PD40/}{Produkte haben eine Feature Tabelle}
  \addrow{/PD50/}{Ein Produkt hat ein Hauptbild}
  \addrow{/PD60/}{Ein Produkt hat Medien in einer Galerie} 
  \addrow{/PD70/}{Anbieter haben mehrere Produkte}
  \addrow{/PD80/}{Anbieter haben einen Namen}
  \addrow{/PD90/}{Anbieter haben eine Kurzbeschreibung}
  \addrow{/PD100/}{Anbieter haben eine Detailbeschreibung}
  \addrow{/PD110/}{Anbieter haben eine Feature Tabelle}
  \addrow{/PD120/}{Ein Anbieter hat ein Hauptbild}
  \addrow{/PD130/}{Ein Anbieter hat Medien in einer Galerie} 
  \addrow{/PD140/}{Ein Feature in der Feature Tabelle hat einen Titel und eine Beschreibung}
  \addrow{/PD150/}{Eine Feature Tabelle hat mehrere Features}
  \addrow{/PD160/}{Medien in der Galerie können PDFs oder Bilder sein}
  \addrow{/PD170/}{Ein User hat einen Namen und ein Passwort}
\end{usecase}

\section{Nichtfunktionale Anforderungen}
\begin{usecase}
  \addheading{Nummer}{Beschreibung} 
  \addrow{/NF10/}{asdf asdf asdf asdf asdf asdf }
  \addrow{/NF20/}{ asdf asdf asdftergfgasdgewr asdfh}
\end{usecase}

\section{Systemmodelle}
\subsection{Datenmodell}
\subsection{Benutzeroberfläche}


\clearpage
%%%%%%%%%%%%%%%%%%%%%%%%%%%%%%%%%%%%%%%%%%%%%%%%%%%%%%%%%%%%%%%%%%%%%%
% Begriffslexikon zur Beschreibung des Produkts						 %
%%%%%%%%%%%%%%%%%%%%%%%%%%%%%%%%%%%%%%%%%%%%%%%%%%%%%%%%%%%%%%%%%%%%%%
\newglossaryentry{anbieter}
{
  name=Anbieter,
  description={Telefon-Anbieter-Firma, die das System nutzt. Der Anbieter trägt die Daten und Informationen seiner Firma und seiner Produkte ein. \\ Synonym: Kunde}
}

\newglossaryentry{bildergalerie}
{
  name=Bildergalerie,
  description={Der User kann Bilder und Dateien der Formate .jpg, .gif, .png oder .pdf in die Bildergalerie für ein Produkt oder den Anbieter hochladen. Außerdem kann er die Dateien in der Liste nach oben oder unten schieben, um die Reihenfolge zu verändern. Die Bildergalerie wird auf der Seite unten mittig als Slidebar angezeigt. Die maximale Anzahl wird über die Speicherkapazität geregelt. \\ Querverweis: Produkt, Anbieter}
}

\newglossaryentry{endverbraucher}
{
  name=Endverbraucher,
  description={Bedeutung: Kunde des Kunden, der das System nutzt, um sich über die verschiedenen Anbieter und deren Produkte zu informieren. \\ Abgrenzung: User, Anbieter \\ Querverweis: Anbieter, Produkt}
}

\newglossaryentry{feature-tabelle}
{
  name=Feature-Tabelle,
  description={Wird von dem User erstellt. Der User kann zu jedem Produkt und Anbieter eine Tabelle mit Produkteigenschaften und einer kurzen Beschreibung anlegen. Mit einem "Plus"-Button kann er neue Zeilen hinzufügen. Die Feature-Tabelle wird auf der Seite unten links angezeigt. \\ Querverweis: Produkt, Anbieter}
}

\newglossaryentry{hauptbild}
{
  name=Hauptbild,
  description={Das Hauptbild wird vom Anbieter aus der Bildergalerie gewählt. Es wird größer angezeigt als die übrigen Bilder der Seite. Es kann die Fromate .jpg, .gif oder .png haben. \\ Querverweis: Bildergalerie, Anbieter}
}

\newglossaryentry{kurzbeschreibung}
{
  name=Kurzbeschreibung,
  description={Kurzer Text der das Produkt oder den Anbieter beschreibt. Die Länge ist voreingestellt. Sie wird von dem User eingetragen und nach dem Speicher in einem Textfeld auf der Seite des Anbieters oder des Produktes angezeigt. \\ Querverweis: User, Anbieter, Produkt}
}

\newglossaryentry{langer-text}
{
  name=Langer Text,
  description={Ausführliche Beschreibung des Produktes oder des Anbieters. Er wird vom User eingetragen und nach dem Speicher in einem Textfeld auf der Seite des Anbieters oder des Produktes angezeigt. \\ Synonym: Detailbeschreibung \\ Querverweis: Produkt, Anbieter}
}

\newglossaryentry{login}
{
  name=Login,
  description={Eingabe des Benutzernamens und des Passwortes durch den User, um im Anschluss die Daten der Produkte und des Anbieters bearbeiten zu können. \\ Querverweis: User, Produkt, Anbieter}
}

\newglossaryentry{logout}
{
  name=Logout,
  description={Abmelden des Users. Falls der User seine Änderungen vor dem Logout nicht gespeichert hat, wird er gefragt, ob er speichern oder zurücksetzen will. \\ Querverweis: User, Zurücksetzten}
}

\newglossaryentry{name}
{
  name=Name,
  description={Name des Anbieters oder des Produktes, der als nicht ändererbarer Text oben auf der Seite des Anbieters angezeigt wird. \\ Querverweis: Anbieter, Produkt}
}

\newglossaryentry{produkt}
{
  name=Produkt,
  description={Ein Produkt ist immer einem Anbieter zugeordnet. Es werden die Informationen über das Produkt von dem Anbieter auf der Seite des Produktes eingetragen. \\ Querverweis: Anbieter}
}

\newglossaryentry{produkt-menu}
{
  name=Produktmenü,
  description={Auf der linken Seite der User-Ansicht beifindet sich eine Auflistung der Produkte des Anbieters. Durch Anklicken, kann ein Produkt oder auch der Anbiter selbst ausgewählt werden und die Informationen könnnen bearbeitet werden. \\ Querverweis: Produkt, Anbieter}
}

\newglossaryentry{seite}
{
  name=Seite,
  description={Ansicht eines Produktes oder Anbieters, die der Endverbraucher hat. \\ Querverweis: Endverbraucher, Produkt, Anbieter}
}

\newglossaryentry{loginResponse}
{
  name=User,
  description={Angestellter des Anbieters, der die Daten und Informationen einträgt. Dieser Angestellte muss über administrative Rechte verfügen, um sich einzuloggen und die Seite zu bearbeiten. \\ Abgrenzung: User, Endverbraucher \\ Querverweis: Anbieter \\ Synonym: Benutzer}
}

\newglossaryentry{loginResponse-ansicht}
{
  name=User-Ansicht,
  description={Ansicht der Bearbeitungsmaske, die der User nutzt, um die Daten des Produktes oder des Anbieters einzutragen. \\ Querverweis: User, Produkt, Anbieter}
}

\newglossaryentry{zurucksetzen}
{
  name=Zurücksetzen,
  description={Wenn der User die Daten eingetragen hat und noch nicht gespeichert hat, können seine Änderungen über den Button "Zurücksetzten", wieder gelöscht werden. Das heißt, es wird der Stand des letzten Speicherns wiederhergestellt. Wenn der User die Seite verlassen will, ohne das er zuvor gespeichert hat, wird er gefragt, ob er speichern oder zurücksetzten möchte. \\ Querverweis: User}
}

%detailbeschreibung langtext synonym
 
% Setze den richtigen Namen für das Glossar
%\renewcommand*{\glossaryname}{\section{\glossarName}}

% Drucke das gesamte Glossar
\glsaddall
\printglossaries

% Trage das Glossar in das Inhaltsverzeichnis ein
\stepcounter{section}
\addcontentsline{toc}{section}{\numberline {\thesection} \glossarName}
 
\end{document}
